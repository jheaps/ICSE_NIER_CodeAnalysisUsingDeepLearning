\section{Evaluation}

The substantial number of Android apps in the Google Play store and open source software repositories provide us with a large dataset for the training of a classification model and the evaluation of our approach. To evaluate our approach, we need to apply it on a concrete application scenario, so that we have a known set of high-level concepts and keywords. We plan to evaluate our approach first on the mapping of information types to code elements in Android security and privacy analysis. Existing works~\cite{rasthofer2014machine}~\cite{huang2015supor} have enumerated a number of high-level concepts and keywords for information types that Android apps collect. 

The major challenge is to obtain ground truth of privacy-related code elements for both training and evaluation purposes. In particular, as mentioned earlier, we plan to use most explicit code elements for training purposes. For evaluation, we plan to manually check the retrieved code elements for each high-level concept. Besides manual confirmation of the results, we plan to further execute the app and track the flow of privacy data (both from user inputs and Android APIs) with taints, so that we can confirm whether a variable holds privacy data or a method processes privacy data at runtime.

%It is also important to evaluate whether our mappings are useful for developers. We will implement our approach as a plug-in of Android Studio which recommends privacy-related facts to developers. With training materials on how to handle each type of privacy code entity we can further help developers find potential bad coding practices regarding privacy. We plan to evaluate the usability of the plugin with human developer subjects, and apply our plugin to open source Android projects and report potential privacy-related coding issues to developers.
