\section{Conclusion and Future Work}

We have described a novel approach to utilize deep learning techniques to assist in the analysis of source code by translating words in source code to word embeddings, using those embeddings as input to a deep learning model, and evaluating the effectiveness and feasability of the approach on privacy detection. Besides providing semantics support to static analysis tools, our approach may enable and lead to a large variety of research projects and techniques in software engineering. First, the mapping from code elements to high level concepts can help developers better understand their code by automatically adding comments and commit messages, or by retrieving relevant code in code search. Second, such automatically generated mappings may facilitate the reuse of code and patterns across project boundary, as they provide an intermediate point (i.e., textual keywords representing high level concepts) to map code elements from different projects. Third, nowadays software is used more and more common in various embedded environments (e.g., smart cars, medical devices, etc.), and they need to follow privacy and safety regulations specific to those areas. The mappings are able to link high level concepts in the regulations to concrete code, and thus enable automatic enforcement of regulations for embedded software.

%\textbf{More advanced representation of code context.}

%\textbf{Application Scenarios.}


